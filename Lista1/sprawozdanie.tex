\documentclass[a4paper,14pt]{report}
\usepackage[utf8]{inputenc}
\usepackage[T1]{fontenc}
\usepackage{titlesec}
\titleformat{\chapter}[display]
  {\normalfont\bfseries}{}{0pt}{\Huge}



\title{Wprowadzenie do sztucznej inteligencji: Lista 1}
\author{Radosław Wojtczak}
\date{numer indeksu: 254607}
\begin{document}
\maketitle
\section{Problematyka}
Problematyką zadania było zastosowanie algorytmu A* (A-gwiazdka) do rozwiązania popularnej gry "Piętnastka". Piętnastka to układanka zbudowana z pudełka, w którym znajduje się 15 kwadratowych klocków o jednakowych rozmiarach ułożonych w kwadrat 4×4 i ponumerowanych od 1 do 15. Jedno miejsce jest puste i umożliwia przesuwanie sąsiednich klocków względem siebie. Oczywiście to pojęcie rozszerza się na więcej wymiarów, co również zostało wzięte pod uwagę w stworzonym przeze mnie programie.
Dodatkowym wymaganiem było użycie dwóch róznych funkcji heurystycznych.

\section{Specyfikacja implementacji}
Program został napisany w pythonie z wykorzystaniem dwóch klas:
\begin{enumerate}
	\item Klasa "Fifteen" zajmująca się przeprowadzaniem rozgrywki, wykonywaniem wszystkich niezbędnch obliczeń oraz przedstawianiem niezbędnych komunikatów. Plansza jest przedstawiona przy pomocy tablicy jednowymiarowej (początkowo chciałem wykonać zadanie przy pomocy tablicy dwuwymiarowej, jednakże czas wykonywania programu, prawdopodobnie ze względu na błędy implementacyjne, był zdecydowanie za duży). Pusty klocek oznaczony jest ostatnim numerem (szerokość*wysokosc).
	\item Klasa "Node" przedstawiająca pojedynczy stan planszy (w terminologii grafów reprezentuje węzeł, stąd nazwa). Zawiera takie informacje jak wartość funkcji heurystycznej dla danego stanu, numer ruchu oraz odwołanie do stanu, z którego do niego przybyliśmy (funkcjonalność wykorzystywana do przedstawienia drogi jaką trzeba pokonać od stanu startowego do ułożenia planszy)
\end{enumerate}
Funkcjonalność programu została rozszerzona o rozwiązywanie stanu gry nawet w sytuacjach, gdy puste pole nie znajduje się w prawym dolnym rogu. \\
Ze względów objętościowych powyższy opis przedstawia jedynie zarys programu. Szczegółowy opis wszystkich specyfikacji został pominięty i zostanie przedstawiony w trakcie oddawania programu.


\section{Generowanie permutacji}
W programie za generowanie permutacji są odpowiedzialne dwie funkcje: \textit{makePermutation()} i \textit{checkPermutation()}. Pierwsza z nich kreuje plansze, druga natomiast dokonuje losowań aż do momentu, gdy plansza będzie rozwiązywalna. \\
Kiedy plansza jest rozwiązywalna? \\
\textbf{INWERSJA}- Inwersją w tym przypadku nazywamy parę liczb $a_{i} , a_{j} $ taką, że $i<j $ jeżeli  $ a_{i} > a{j} $ \\
i,j-indeksy w tabeli
\begin{itemize}
	\item Jeśli szerokość planszy jest liczbą nieparzystą to liczba inwersji jest liczbą parzystą.
	\item Jeśli szerokość jest liczbą parzystą i pusty kafelek znajduje się w nieparzystym rzędzie licząc od dołu to liczba inwersji musi być liczbą parzystą
	\item Jeśli szerokość jest liczbą parzystą i pusty kafelek znajduje się w parzystym rzędzie licząc od dołu, to liczba inwersji musi być liczbą nieparzystą.
\end{itemize}



\section{Użyte heurystyki}
\begin{enumerate}
	\item Heurystyka określona w programie jako \textit{distance\_heuristic()} to heurystyka, która dla każdego kafelka zwraca jego odległość względem spodziewanego końcowego położenia. Odległość oznacza sumę kafelków w pionie jak i w poziomie, jaką kafelek musi pokonać aby dotrzeć do celu. Nie bierzemy pod uwagę ruchów "na ukos", gdyż ów ruchy w tej grze są nielegalne. Wynikiem zwracanym przez powyższą funkcję jest wysumowana odległość.
	\item Heurystyka określona w programie jako \textit{linear\_heuristic()} to lekko rozbudowana wersja poprzedniej heurystyki. Do wcześniej obiczonej wartości (otrzymanej przez funkcję \textit{distance\_heuristic()}) dodajemy jeszcze liczbę "liniowych konfliktów" występujących między kafelkami. Jako liniowy konflikt uznajemy sytuację, gdy dwa kafelki znajdują się w tym samym rzędzie (kolumnie) i miejsca docelowe znajdują się w tym samym rzędzie (kolumnie) jednak jeden z kafelków przeszkadza drugiemu w dojściu do celu. Do rozwiązania tego konfliku potrzeba dwóch ruchów (w pierwszym kafelek przeszkadzający musi ustąpić miejsca, w drugim rozpatrywany kafelek zajmuje swoje miejsce), także liczbę występujących konfliktów mnożymy razy 2.
	Ostatecznie otrzymujemy wzór
	\begin{equation}
		\textit{linear\_heuristic()} = \textit{distance\_heuristic()}  + 2 * liczba\_konfliktów
	\end{equation}
\end{enumerate}
W pliku znajduje się również implementacja heurystyki \textit{misplaced\_heuristic()}, jednakże ze względu na bardzo słabe wyniki nie będzie ona analizowana w dalszej części sprawozdania.


\section{Wyniki i interpetacja}
\textbf{UWAGA}: Plansze przedstawione w formie tablicy jednowymiarowej.\\
Kolejność: od góry do dołu, od lewej do prawej \\
Poniższe tabele przedstawiają wyniki dla wykonanych testów:
\begin{enumerate}
	\item Liczba odwiedzonych stanów
	\item Liczba wykonanych przesunięć
	\item Czas
\end{enumerate}
\begin{table}[h!]
\centering
\begin{tabular}{|c | c | c | c|} 
 \hline
 Plansza & \textit{distance\_heuristic()} & \textit{linear\_heuristic()} \\[0.5ex] 
 \hline\hline
 $[14,5,12,3,13,1,9,11,16,7,8,4,2,10,6,15]$ & 592 552 & 399 793 \\ 
 $[1,3,16,15,5,7,11,4,9,2,8,12,13,10,14,6]$ & 11 273 & 8 025 \\
 $[2,5,1,3,6,8,11,12,16,9,7,14,13,10,15,4]$ & 30 751 & 7 007 \\
 $[13,1,2,10,5,3,11,7,9,6,16,8,14,15,4,12]$ & 87 411 & 15 520 \\
 $[14,1,2,8,5,6,4,3,9,7,11,10,13,12,16,15]$ & 10 099 & 3 683 \\
 $[8,1,7,3,13,10,6,5,2,15,14,4,9,16,11,12]$ & 143 005 & 40 519 \\
 $[12,14,8,9,6,4,2,10,15,16,3,11,5,13,1,7]$ & 3 018 170 & 1 480 419 \\
 $[14,13,2,11,4,6,5,10,3,1,16,8,7,15,9,12]$ & 1 906 763 & 552 249 \\
 $[2,10,7,3,1,5,6,9,13,14,11,4,16,15,12,8]$ & 13 886 & 5 933 \\
 $[5,8,6,12,10,16,13,14,3,9,1,2,4,11,15,7]$ & 507 118 & 231 267 \\
 $[4,2,3,7,1,6,12,8,15,9,10,11,16,5,13,14]$ & 252 538 & 7 565 \\
 $[16,7,1,4,3,10,11,8,5,15,14,6,2,13,9,12]$ & 55 523 & 28 150 \\
 Wartoścć średnia: & 552 424& 231 677\\

 \hline
\end{tabular}
\caption{Liczba odwiedzonych stanów}
\label{Stany}
\end{table}

\begin{table}[h!]
\centering
\begin{tabular}{|c | c | c | c|} 
 \hline
 Plansza & \textit{distance\_heuristic()} & \textit{linear\_heuristic()} \\[0.5ex] 
 \hline\hline
 $[14,5,12,3,13,1,9,11,16,7,8,4,2,10,6,15]$ & 46 & 46 \\ 
 $[1,3,16,15,5,7,11,4,9,2,8,12,13,10,14,6]$ & 30 & 32 \\
 $[2,5,1,3,6,8,11,12,16,9,7,14,13,10,15,4]$ & 34 & 34 \\
 $[13,1,2,10,5,3,11,7,9,6,16,8,14,15,4,12]$ & 36 & 36 \\
 $[14,1,2,8,5,6,4,3,9,7,11,10,13,12,16,15]$ & 31 & 31 \\
 $[8,1,7,3,13,10,6,5,2,15,14,4,9,16,11,12]$ & 40 & 40 \\
 $[12,14,8,9,6,4,2,10,15,16,3,11,5,13,1,7]$ & 55 & 55 \\
 $[14,13,2,11,4,6,5,10,3,1,16,8,7,15,9,12]$ & 52 & 52 \\
 $[2,10,7,3,1,5,6,9,13,14,11,4,16,15,12,8]$ & 33 & 33 \\
 $[5,8,6,12,10,16,13,14,3,9,1,2,4,11,15,7]$ & 52 & 52 \\
 $[4,2,3,7,1,6,12,8,15,9,10,11,16,5,13,14]$ & 39 & 37 \\
 $[16,7,1,4,3,10,11,8,5,15,14,6,2,13,9,12]$ & 38 & 38 \\
 Wartość średnia: & 40.5 & 40.5\\ 

 \hline
\end{tabular}
\caption{Liczba kroków potrzebna do zwycięstwa}
\label{Kroki}
\end{table}

\begin{table}[h!]
\centering
\begin{tabular}{|c | c | c | c|} 
 \hline
 Plansza & \textit{distance\_heuristic()} & \textit{linear\_heuristic()} \\[0.5ex] 
 \hline\hline
 $[14,5,12,3,13,1,9,11,16,7,8,4,2,10,6,15]$ & 46.89s & 112.1s \\ 
 $[1,3,16,15,5,7,11,4,9,2,8,12,13,10,14,6]$ & 0.79s & 2.10s \\
 $[2,5,1,3,6,8,11,12,16,9,7,14,13,10,15,4]$ & 2.15s & 1.65s \\
 $[13,1,2,10,5,3,11,7,9,6,16,8,14,15,4,12]$ & 7.24s & 3.81s \\
 $[14,1,2,8,5,6,4,3,9,7,11,10,13,12,16,15]$ & 0.71s & 0.93s \\
 $[8,1,7,3,13,10,6,5,2,15,14,4,9,16,11,12]$ & 11.35s & 11.01s \\
 $[12,14,8,9,6,4,2,10,15,16,3,11,5,13,1,7]$ & 255.0s & 421.1s \\
 $[14,13,2,11,4,6,5,10,3,1,16,8,7,15,9,12]$ & 152.9s & 149.5s \\
 $[2,10,7,3,1,5,6,9,13,14,11,4,16,15,12,8]$ & 0.96s & 1.71s \\
 $[5,8,6,12,10,16,13,14,3,9,1,2,4,11,15,7]$ & 42.3s & 62.6s \\
 $[4,2,3,7,1,6,12,8,15,9,10,11,16,5,13,14]$ & 20.7s & 1,82s \\
 $[16,7,1,4,3,10,11,8,5,15,14,6,2,13,9,12]$ & 4.64s & 7.43s \\
 Wartość średnia: & 45.5s & 64.5s\\

 \hline
\end{tabular}
\caption{Czas potrzebny do ułożenia układanki}
\label{Czas}
\end{table}
\textbf{UWAGA}: Wyniki otrzymane w tabeli \ref{Czas} mogą nie oddawać rzeczywistej szybkości wykonywania się powyższych algorytmów. Może być to spowodowane miedzy innymi wykonywaniem innych czynności na komputerze w trakcie pracy algorytmu.\\
\textbf{Interpretacja}: Otrzymane wyniki pokazują, iż wiele zmiennych wpływa na wydajność działania programu. Przede wszystkim otrzymane wyniki w tabelach \textbf{\ref{Czas}} i \textbf{\ref{Stany}} pokazują, iż użyta heurstyka ma spore znaczenie i dobór odpowiedniej w znacznym stopniu może przyśpieszyć pracę naszego programu. Z rozpatrywanych heurystyk zauważamy, iż mimo mniejszej liczby odwiedzonych stanów \textit{linear\_heuristic} średnio potrzebuje więcej czasu na rozwiązanie układanki, jednakże trzeba wziąć poprawkę na to, iż inny sposób implementacyjny może wpłynąć na osiągi powyższych heurystyk. \\
Dodatkowo zauważamy, że średnie wartości potrzebnych ruchów do rozwiązania układanki są równe, bliskie liczbie 40. Biorąc pod uwagę, iż górna granica liczby ruchów dla układanki o wymiarach 4x4 to 80, stwierdzenie, iż powyższe heurystyki zostały odpowiednio zaimplementowane wydaje się być poprawne. Ponadto w żadnym pojedynczym układzie startowym liczba ruchów nie przekroczyła wyżej wspomnianej liczby 80.




\end{document}