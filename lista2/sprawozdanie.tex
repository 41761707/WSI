\documentclass[a4paper,14pt]{report}
\usepackage[utf8]{inputenc}
\usepackage[T1]{fontenc}
\title{Wprowadzenie do sztucznej inteligencji, Lista 2}
\author{Radosław Wojtczak}
\date{254607}
\begin{document}
\maketitle
\section{Wprowadzenie}
	Celem zadania było wykorzystanie zbioru danych \textbf{THE MNIST DATABASE of handwritten digits} i przy pomocy biblioteki \textbf{TensorFlow} należało stworzyć i wytrenować sieć neuronową rozpoznającą cyfry z podanego zbioru.
\section{Rozwiązanie}
	Ów program został napisany w języku programowania python, przy pomocy narzędzia "Google Colab".\\
	Na samym początkowy dokonujemy importu wszystkich potrzebnych bibliotek, w tym wcześniej wspomnianego \textbf{TensorFlow'a}. Przygotowujemy odpowiednio dane i dzielimy je na dwie grupy:
	\begin{itemize}
		\item \textit{train}- odpowiedzialne za trenowanie sieci neuronowej
		\item \textit{test}- testowy zbiór danych.
	\end{itemize}
	Będziemy pracować według następującego przepływu roboczego: najpierw zajmiemy się treningiem sieci neuronowej na danych treningowych, następnie sieć nauczy się kojarzyć obrazi z etykietami i wygeneruje swoje przewidywania dotyczące zbioru test. 
	Dodatkowo sprawdzimy jak powyższa sieć poradzi sobie z specjalną serią danych przygotowaną przez autora tego sprawozdania, jak i innej osoby trzeciej. \\
	Utworzona sieć składa się z sekwencji trzech warstw \textit{Dense}, które są ze sobą połączone w sposób gęsty. Trzecia z warstw jest dziesięcioelementową warstwą \textit{softmax}- warstwa ta zwróci tablicę 10 wartości prawdopodobieństwa (suma wszystkich tych wartości jest równa 1). Każdy z tych wyników określa prawdopodobieństwo tego, że na danym obrazie przedstawiono daną cyfrę (obraz może przedstawiać jedną z dostępnych dziesięciu cyfr).
	Dodatkowo na etapie kompilacji dochodzi do ustalnie odpowiedniej funkcji straty, optymalizatora oraz metryki monitorowania podczas trenowania i testowania (za to wszystko odpowiada wywołanie \textit{model.compile}). \\
	Następnie przeprowadzamy trenowanie sieci i sprawdzamy jak zareaguje na testowe dane.
\section{Otrzymane wyniki dla danych z pakietu}
	Po wykonaniu 3 \textit{Epochów} nasza sieć osiągnęła wynik w okolicach \textbf{98,5\%} dokładności dla danych treningowych (Uwaga: każde uruchomienie sieci niezacznie wpływa na zmianę ów dokładności), jednakże zauważamy, że ta wartość dla danych testowych jest mniejsza: wynosi \textbf{97,5\%}. Wynika to między innymi z \textbf{nadmiernego dopasowania} (modele uczenia maszynowego mają tendencję do niższej dokładności przetwarzania nowych danych, niż miało to miejsce w przypadku danych treningowych). Mimo wszystko zauważamy, że ów wartość jest na bardzo zadowalającym poziomie.
\section{Otrzymane wyniki dla danych własnych}
	Poza danymi z wcześniej wspomnianego pakietu sieć została sprawdzona przy użyciu trzech dodatkowych zestawów danych (każdy z nich składał się z 5 powtórzeń każdej z cyfr). W skład tych pakietów wchodziły:
	\begin{enumerate}
		\item Odręcznie napisane cyfry przez autora sprawozdania
		\item Cyfry napisane przy użyciu programu graficznego przez autora sprawozdania
		\item Odręcznie napisane cyfry przez osobę trzecią
	\end{enumerate}
	Każdy z powyższych zestawów znajduje się w plikach źródłowych załączonych wraz ze sprawozdaniem.
	Poniższe tabele przedstawiają otrzymane wyniki dla powyższych danych:
	\begin{table}[h!]
	\centering
	\begin{tabular}{|c | c | c |}
	 \hline
	 Liczba & Uzyskane wyniki & Procent dokładności \\
	 \hline\hline
	 0 & $\{5,3,2,5,3\}$ &  $0\%$ \\
	 1 & $\{1,1,1,3,1\}$ & $80\%$   \\
	 2 & $\{2,2,2,6,3\}$ & $60\%$   \\
	 3 & $\{3,3,3,3,3\}$ & $100\%$   \\
	 4 & $\{6,8,5,8,6\}$ & $0\%$   \\
	 5 & $\{5,3,5,5,3\}$ & $60\%$   \\
	 6 & $\{5,6,6,5,2\}$ & $40\%$   \\
	 7 & $\{3,2,2,8,3\}$ & $0\%$   \\
	 8 & $\{3,3,5,8,3\}$ & $20\%$   \\
	 9 & $\{8,3,9,9,8\}$ & $40\%$   \\
	 \hline
	\end{tabular}
	\caption{Przykładowe wyniki dla odręcznych cyfr autora, wartość średnia:38\%}
	\label{WynikiOdreczneAutor}
	\end{table}

	\begin{table}[h!]
	\centering
	\begin{tabular}{|c | c | c |}
	 \hline
	 Liczba & Uzyskane wyniki & Procent dokładności \\
	 \hline\hline
	 0 & $\{6,6,5,0,6\}$ &  $20\%$ \\
	 1 & $\{1,5,3,3,1\}$ & $40\%$   \\
	 2 & $\{2,2,3,2,2\}$ & $80\%$   \\
	 3 & $\{6,3,3,3,3\}$ & $80\%$   \\
	 4 & $\{6,6,5,6,6\}$ & $0\%$   \\
	 5 & $\{6,5,5,5,5\}$ & $80\%$   \\
	 6 & $\{3,6,5,6,5\}$ & $40\%$   \\
	 7 & $\{6,2,8,2,8\}$ & $0\%$   \\
	 8 & $\{8,8,8,8,8\}$ & $100\%$   \\
	 9 & $\{3,5,3,3,3\}$ & $0\%$   \\
	 \hline
	\end{tabular}
	\caption{Przykładowe wyniki dla "komputerowo napisanych" cyfr autora, wartość średnia:44\%}
	\label{WynikiKompAutor}
	\end{table}

	\begin{table}[h!]
	\centering
	\begin{tabular}{|c | c | c |}
	 \hline
	 Liczba & Uzyskane wyniki & Procent dokładności \\
	 \hline\hline
	 0 & $\{0,0,6,3,6\}$ &  $40\%$ \\
	 1 & $\{8,3,8,8,1\}$ & $20\%$   \\
	 2 & $\{8,2,3,2,2\}$ & $60\%$   \\
	 3 & $\{3,6,3,3,3\}$ & $80\%$   \\
	 4 & $\{1,3,7,8,7\}$ & $0\%$   \\
	 5 & $\{5,5,5,5,6\}$ & $80\%$   \\
	 6 & $\{6,6,6,5,6\}$ & $80\%$   \\
	 7 & $\{3,3,7,5,3\}$ & $20\%$   \\
	 8 & $\{5,3,3,3,3\}$ & $0\%$   \\
	 9 & $\{3,2,3,3,3\}$ & $0\%$   \\
	 \hline
	\end{tabular}
	\caption{Przykładowe wyniki dla odręcznie napisanych cyfr osoby trzeciej, wartość średnia:38\%}
	\label{WynikiOdrecznieSiostra}
	\end{table}
	\textbf{UWAGA:} Kolumna \textbf{Uzyskane wyniki} przedstawia zbiór odpowiedzi sieci na daną liczbę. Każda z liczb została powtórzona pięc razy, dlatego wynikiem jest zbiór zawierający pięć różnych odpowiedzi sieci. Dodatkowo otrzymane wyniki z każdym uruchomieniem lekko się od siebie różnia, w tabelach są przedstawione przykładowe wyniki dla jednego uruchomienia
\section{Interpretacja wyników i wnioski}
	Zauważamy, iż dla naszych danych sieć poradziła sobie bardzo słabo, procentowa wartość dokładności mieściła się w przedziale $[40,45]\%$ dla większości wykonanych testów. Tak masywna różnica w otrzymanych rezultatach wynika z różnic w zapisie cyfr między Europejczykami a Amerykanami. Najłatwiej można to zaobserwować na przykładzie cyfry \textbf{4}, którą sieć w większości przypadków myli z \textbf{6}, czy z \textbf{9}, która jest nagminnie mylona z cyfrą \textbf{3}. Istnieją również przypadki odwrotne, z którymi nasza sieć radzi sobie z bardzo dużą skutecznością- mowa tu o cyfrach takich jak \textbf{3}, \textbf{5}, czy \textbf{2}. \\
	\textbf{Wniosek:} Przy trenowania sieciu neuronowych należy uważnie wziąć pod uwagę dane, które będą wykorzystywane do jej trenowania jak i testowania. Przy danych do trenowania, które będą pochodzić np. z innych kręgów kulturowych niż dane do testowania, wytrenowana sieć neuronowa może mieć spory problem z poprawnym dopasowaniem otrzymanych danych względem trenowanych wzorców. Dodatkowo trzeba uważać na takie zjawiska jak \textbf{nadmierne dopasowanie}, które również negatywnie wpływają na wyniki osiągane przez sieci neuronowe
\end{document}